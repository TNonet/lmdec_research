\documentclass[12pt]{article}
\usepackage[utf8]{inputenc}
\usepackage{graphicx}
\usepackage{amsmath}
\usepackage[a4paper,width=150mm,top=25mm,bottom=25mm]{geometry}

\title{
{Sparse Matrix Storage and Operations}
}



\author{Tim Nonet \\ Supervised by: Dr. Rahul Mazumder}

\begin{document}

\maketitle
https://www.mathcha.io/editor
\begin{abstract}
The main idea of this project is to create user friendly implementations of sparse matrices and their basic operations to allow for efficient storage and computation of SNP data.

\end{abstract}
\section{Introduction}
Storing many specific data sets, such as SNP, as dense matrices is quite inefficient due to the discrete and skewed distribution of entries in the data set. 
\subsection{Compressed Row Format (CSR) }
Depending on the specific structure of a sparse matrix different formats for efficiently storage exist. In addition, the anticipated use of the matrix can make certain formats more beneficial than others \cite{sparskit}. Our need for fast matrix vector multiplication suggest the use of CSR format. However, the need for the transpose operation adds some disadvantages to CSR. \newline \newline

\noindent
\textbf{Advantages}
\begin{enumerate}
\item Extremely quick matrix-vector multiplication
\item Straight forward multi-threading 
\item Easy extension to parallel GPU/CPU calculations 
\item Linear storage in number of non-zero elements 
\end{enumerate}

\noindent
\textbf{Disadvantages} 
\begin{enumerate}
\item Slow and memory expensive transpose operation
\item Non-negligible Initialization Cost 
\item Expensive to alter or update matrix
\end{enumerate}

\subsubsection{CSR Definition}
The standard CSR format is defined by four parameters.
\begin{enumerate}
\item Row Pointer: A list of integers that define how many non-zero values are in each row
\item Column Index: A list of integers that define which items in each row are non-zero
\item Data List: A list of the non-zero values seen in the matrix
\item Shape Tuple: The shape of the matrix
\end{enumerate} 

\noindent
\textbf{Row Pointer}
A list of integers of length equal to the number of rows in the matrix plus 1. If $R_p$ is a row point for a matrix then $R_p[i+1]$ will give the index at which the $i$-th row starts in the Column Index list. This allows for quick access to a specific row of the matrix.
\begin{center}
	\begin{equation}
		\begin{split}
			R_p[0] = 0\\
			R_p[i+1] = ||a_i||_0 + R_p[i] 
		\end{split}
	\end{equation}
\end{center}
\noindent
\textbf{Column Index}
A list of integers of length equal to the number of non-zero elements in the matrix. If $C_i$ is a column index for a matrix then $C_i[R_p[j]:R_p[j+1]]$ will be the column indexes for the non-zero elements in row $j$.
\begin{center}
	\begin{equation}
		\begin{split}
			C_i[R_p[i+1]:R_p[i]] = \{j|a_{ij}  \neq 0\}
		\end{split}
	\end{equation}
\end{center}
\noindent
\textbf{Data List}
A list of the specified non-zero elements stored in matrix. If $D_l$ is a data list for a matrix then $D_l[R_p[j]:R_p[j+1]]$ will be the list of non-zero elements in row $j$  in the same order as $C_i[R_p[j]:R_p[j+1]]$.
\begin{center}
	\begin{equation}
		\begin{split}
			D_l[R_p[i+1]:R_p[i]] = \{a_{ij} |a_{ij}  \neq 0\}
		\end{split}
	\end{equation}
\end{center}
\noindent
\textbf{Shape Tuple}
A 2-tuple of the shape of the matrix. (Number of Rows, Number of Columns) \newline \newline
\noindent
\subsubsection{Example}
Given a densely stored matrix:
\begin{center}
\begin{equation*}
	A=
	\begin{bmatrix}
		2&2&0&1\\
		3&3&4&1\\
		0&0&0&0\\
		0&0&-1&2
	\end{bmatrix}
\end{equation*}
\end{center}
The CSR format for \textbf{Row Pointer}, \textbf{Column Index}, \textbf{Data List} and \textbf{Shape Tuple} are constructed as follows: \newline \newline
\noindent
\textbf{Row Pointer}
\begin{center}
\begin{equation*}
\begin{bmatrix}
2 & 2 & 0 & 1\\
3 & 3 & 4 & 1\\
0 & 0 & 0 & 0\\
0 & 0 & -1 & 2
\end{bmatrix}\xrightarrow[\text{}]{\text{Rows}}\begin{matrix}
( \ 2 & 2 & 0 & 1\ )\\
( \ 3 & 3 & 4 & 1\ )\\
( \ 0 & 0 & 0 & 0\ )\\
( \ 0 & 0 & -1 & 2\ )
\end{matrix}\xrightarrow[\text{Counts}]{\text{Non Zero}}\begin{matrix}
3\\
4\\
0\\
2
\end{matrix}\xrightarrow[\text{Sum}]{\text{Cumulative}}\begin{matrix}
R_{p}[ 0] \ =\ 0 & \rightarrow  & 0 &  & \\
R_{p}[ 1] \ =\ R_{p}[ 0] \ +\ ||a_{1} ||_{0} & \rightarrow  & 3 &  & \\
R_{p}[ 2] \ =\ R_{p}[ 1] \ +||a_{2} ||_{0} & \rightarrow  & 7 &  & \\
R_{p}[ 3] \ =R_{p}[ 2] \ +||a_{3} ||_{0} & \rightarrow  & 7 &  & \\
R_{p}[ 4] \ =R_{p}[ 3] \ +|a_{4} ||_{0} & \rightarrow  & 9 &  & 
\end{matrix}
\end{equation*}
\end{center}
\begin{center}
\begin{equation*}
	R_p=
	\begin{bmatrix}
		0\\
		3\\
		7\\
		7\\
		9
	\end{bmatrix}
\end{equation*}
\end{center}

\noindent
\textbf{Column Index}


\begin{equation*}
\begin{bmatrix}
2 & 2 & 0 & 1\\
3 & 3 & 4 & 1\\
0 & 0 & 0 & 0\\
0 & 0 & -1 & 2
\end{bmatrix}\xrightarrow[\text{}]{\text{Rows}}\begin{matrix}
( \ 2 & 2 & 0 & 1\ )\\
( \ 3 & 3 & 4 & 1\ )\\
( \ 0 & 0 & 0 & 0\ )\\
( \ 0 & 0 & -1 & 2\ )
\end{matrix}\xrightarrow[\text{Indices}]{\text{Non Zero}}\begin{matrix}
0 & 1 & 3 & \\
0 & 1 & 2 & 3\\
 &  &  & \\
3 & 4 &  & 
\end{matrix}\xrightarrow{\text{Flatten}}\begin{matrix}
0\\
1\\
3\\
0\\
1\\
2\\
3\\
3\\
4
\end{matrix}
\end{equation*}

\begin{center}
\begin{equation*}
	C_i=
	\begin{bmatrix}
		0\\
		1\\
		3\\
		0\\
		1\\
		2\\
		3\\
		3\\
		4
	\end{bmatrix}
\end{equation*}
\end{center}
\noindent
\textbf{Data List}


\begin{equation*}
\begin{bmatrix}
2 & 2 & 0 & 1\\
3 & 3 & 4 & 1\\
0 & 0 & 0 & 0\\
0 & 0 & -1 & 2
\end{bmatrix}\xrightarrow[\text{}]{\text{Rows}}\begin{matrix}
( \ 2 & 2 & 0 & 1\ )\\
( \ 3 & 3 & 4 & 1\ )\\
( \ 0 & 0 & 0 & 0\ )\\
( \ 0 & 0 & -1 & 2\ )
\end{matrix}\xrightarrow[\text{Values}]{\text{Non Zero}}\begin{matrix}
2 & 2 & 1 & \\
3 & 3 & 4 & 1\\
 &  &  & \\
-1 & 2 &  & 
\end{matrix}\xrightarrow{\text{Flatten}}
\begin{matrix}
2\\
2\\
1\\
3\\
3\\
4\\
1\\
-1\\
2
\end{matrix}
\end{equation*}
\begin{center}
\begin{equation*}
	D_l = 
	\begin{bmatrix}
	2\\
	2\\
	1\\
	3\\
	3\\
	4\\
	1\\
	-1\\
	2
	\end{bmatrix}
\end{equation*}
\end{center}

\noindent
\textbf{Sparse A}
\begin{equation*}
	A=
	\begin{bmatrix}
		2&2&0&1\\
		3&3&4&1\\
		0&0&0&0\\
		0&0&-1&2
	\end{bmatrix} 
	\rightarrow
	R_p=
	\begin{bmatrix}
		0\\
		3\\
		7\\
		7\\
		9
	\end{bmatrix},
		C_i=
	\begin{bmatrix}
		0\\
		1\\
		3\\
		0\\
		1\\
		2\\
		3\\
		3\\
		4
	\end{bmatrix},
	D_l = \begin{bmatrix}
	2\\
	2\\
	1\\
	3\\
	3\\
	4\\
	1\\
	-1\\
	2
	\end{bmatrix},
	S = (4,4)
\end{equation*}

To get the first row.


\begin{gather*}
\{i|a_{1} \neq 0\} \ =\ C_{i}[ R_{p}[ 0] :R_{p}[ 1]] \ =C_{i}[ 0:3] \ =\ \begin{bmatrix}
0\\
1\\
3
\end{bmatrix}\\
\{a_{1i} |a_{1i} \neq 0\} \ =D_{l}[ R_{p}[ 0] :R_{p}[ 1]] \ =\ D_{l}[ 0:3] \ =\begin{bmatrix}
2\\
2\\
1
\end{bmatrix} \ \\
a_{1} \ =\begin{bmatrix}
2 & 2 & 0 & 1
\end{bmatrix} \ 
\end{gather*}
\subsubsection{Memory Cost and Binary Matrix}
If all of the elements of a matrix are known to be the same. It is not necessary to store the Data List, $D_l$ and in stead just store one value. This can decrease the memory cost of storing a matrix.

\noindent
\textbf{Sparse Binary B example}
\begin{equation*}
	B=
	\begin{bmatrix}
		a&a&0&a\\
		a&a&a&a\\
		0&0&0&0\\
		0&0&a&a
	\end{bmatrix} 
	\rightarrow
	R_p=
	\begin{bmatrix}
		0\\
		3\\
		7\\
		7\\
		9
	\end{bmatrix},
		C_i=
	\begin{bmatrix}
		0\\
		1\\
		3\\
		0\\
		1\\
		2\\
		3\\
		3\\
		4
	\end{bmatrix},
	D_l = [a],
	S = (4,4)
\end{equation*}
\noindent
\textbf{Memory Cost}
The memory usage of storing a Sparse CSR matrix is the cost of storing \textbf{Row Pointer}, \textbf{Column Index}, \textbf{Data List} and \textbf{Shape Tuple}. Luckily, textbf{Row Pointer}, \textbf{Column Index} are integers and we can choose between many different types of integers which will give different storage capacities and difference memory costs seen below in the following table: (Notice that all formats are unsigned as indexing is always positive and that each row has on average $K$ entries). Looking at the table we see what drives the needed size of the integer is not the number of non zero elements needed but the number of rows needed.  

\begin{table}[]
\begin{tabular}{|l|l|l|l|}
\hline
Integer Type              & Number of Rows                     & Number of Columns & Number of Non-Zero- \\        
                                    & $N_r$  & $N_c$                        & Elements $N_{nnz}$                             \\ \hline
8-bit Unsigned Int     & 255/K                            &                 255            &            255                              \\ \hline
16-bit Unsigned Int   &  6.5e4/K                                                 &          6.5e4     &  6.5e4                        \\ \hline
32-bit Unsigned Int  &  4.3e9/K                                                  &        4.3e9          &  4.3e9                             \\ \hline
64-bit Unsigned Int  &  1.8e19/K                                                  &   1.8e19   & 1.8e19                 \\ \hline
\end{tabular}
\end{table}




\section{Notation}
This is the

\section{References}
\bibliographystyle{plain}
\bibliography{bib1}

\end{document}
